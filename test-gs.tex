%!TEX TS-program = xelatex
%!TEX encoding = UTF-8 Unicode

\documentclass{gs}

\title{La mia classe personalizzata}
\subtitle{Studio della filosofia greca attraverso LaTeX}
\author{Il mio nome}
\date{\today}

% Definizione delle parole chiave per i metadati PDF
\kuno{filosofia}
\kdue{greco antico}
\ktre{Platone}
\kquattro{etimologia}
\kcinque{XeLaTeX}

\begin{document}

\maketitle

\section{Introduzione}

Il termine \etimologia{filosofia}{greco φιλοσοφία, “amore per la sapienza”} rappresenta la ricerca fondamentale dell'essere umano verso la comprensione del mondo e di se stesso.

\section{Platone e le idee}

Il \concetto{mondo delle idee} platonico si esprime attraverso il greco \greco{κόσμος νοητός} (kosmos noetos), ovvero il mondo intelligibile contrapposto a quello sensibile.

\begin{citazionefilosofica}{Platone, Repubblica VII, 514a}
Εἰκάσαι τοιούτῳ πάθει τὴν ἡμετέραν φύσιν παιδείας τε πέρι καὶ ἀπαιδευσίας.
\end{citazionefilosofica}

Questa celebre allegoria della caverna illustra la condizione dell'uomo nei confronti della conoscenza e dell'educazione.

\subsection{L'episteme e la doxa}

La distinzione tra \concetto{episteme} (ἐπιστήμη, conoscenza scientifica) e \concetto{doxa} (δόξα, opinione) rappresenta uno dei pilastri del pensiero platonico.

\section{Conclusioni}

L'approccio filosofico che considera l'\etimologia{etimologia}{greco ἐτυμολογία, "studio del vero significato"} come strumento di comprensione concettuale si rivela particolarmente fruttuoso nello studio del pensiero antico.

\end{document}
